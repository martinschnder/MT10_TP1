\documentclass[titlepage]{article}
\usepackage[utf8]{inputenc}
\usepackage[T1]{fontenc}
\usepackage{listings}
\usepackage{tabularx}

\title{Rapport MT10 - TP1 : Groupes d’ordre 4}
\author{Martin Schneider, Océane Bordeau}

\setlength{\parindent}{0pt}

\begin{document}
    \maketitle
    \textbf{Question 0 :}

    Définition d'un groupe : Un groupe est un triplet (G, *, e) où M est un ensemble, * une application
    \[*:G \times G \longrightarrow M\]
    \[(x, x') \longrightarrow x*x'\]
    et $e \in G$ tel que : 

    (i) e est un \textbf{élément neutre} : \[(\forall x \in G) \quad x*e=x \quad et \quad e*x=x\]
    (ii) * est \textbf{associative} : \[(\forall x, x', x'' \in G) \quad (x*x')*x''=x*(x'*x'')\]
    (iii) Tout élément de G adment un \textbf{symetrique} : \[(\forall x \in G)(\exists y \in G) \quad x*y=e \quad et \quad y*x=e\]

    Définition d'un morphisme de groupe : Soient (G, *, e) et (G', *', e') deux groupes. 
    Une application de G dans G' est un morphisme de groupe si elle est compatible avec les structures de groupe.
    \[(\forall x, y \in G) \quad f(x*y)=f(x)*'f(y)\]

    \section{Test d'une loi}
        \subsection{Codage d'une loi sur un ensemble}
        \textbf{Question 1 :}

        Groupe $(Z/4Z, +, 0)$
        \[E = \{\dot{0}, \dot{1}, \dot{2}, \dot{3}\}\]

        \[t=\begin{tabular}{| c || c | c | c | c |}
            \hline
            $x$\textbackslash $x'$ & $\dot{0}$ & $\dot{1}$ & $\dot{2}$ & $\dot{3}$ \\ \hline \hline
            $\dot{0}$ & $\dot{0}$ & $\dot{1}$ & $\dot{2}$ & $\dot{3}$ \\ \hline
            $\dot{1}$ & $\dot{1}$ & $\dot{2}$ & $\dot{3}$ & $\dot{0}$ \\ \hline
            $\dot{2}$ & $\dot{2}$ & $\dot{3}$ & $\dot{0}$ & $\dot{1}$ \\ \hline
            $\dot{3}$ & $\dot{3}$ & $\dot{0}$ & $\dot{1}$ & $\dot{2}$ \\
            \hline
        \end{tabular}\]

        Groupe $(Z/2Z \times Z/2Z, +, 0)$
        \[E = \{(0, 0), (1, 0), (0, 1), (1, 1)\}\]
  
        \[t=\begin{tabular}{| c || c | c | c | c |}
            \hline
            $x$\textbackslash $x'$ & (0, 0) & (1, 0) & (0, 1) & (1, 1) \\ \hline \hline
            (0, 0) & (0, 0) & (1, 0) & (0, 1) & (1, 1) \\ \hline
            (1, 0) & (1, 0) & (0, 0) & (1, 1) & (0, 1) \\ \hline
            (0, 1) & (0, 1) & (1, 1) & (0, 0) & (1, 0) \\ \hline
            (1, 1) & (1, 1) & (0, 1) & (1, 0) & (0, 0) \\
            \hline
        \end{tabular}\]

        Avec SageMath :

        \begin{tabularx}{11.5cm}{|p{0.60cm}|X|}
            \hline
            \verb|In|
            & 
            \verb|C4 = groups.permutation.Cyclic(4)|
            \\
            \hline
        \end{tabularx}
            
        \begin{tabularx}{11.5cm}{|p{0.60cm}|X|}
            \hline
            \verb|In|
            & 
            \verb|t_C4 = C4.cayley_table().table()|
            \\
            \hline
        \end{tabularx}

        {\setlength{\arraycolsep}{2ex}
        \[\begin{array}{r|*{4}{r}}
            \multicolumn{1}{c|}{\ast}&a&b&c&d\\\hline
            {}a&a&b&c&d\\
            {}b&b&c&d&a\\
            {}c&c&d&a&b\\
            {}d&d&a&b&c\\
        \end{array}\]}

        \begin{tabularx}{11.5cm}{|p{0.60cm}|X|}
            \hline
            \verb|In|
            & 
            \verb|C2 = groups.permutation.Cyclic(2)|
            \\
            \hline
        \end{tabularx}

        \begin{tabularx}{11.5cm}{|p{0.60cm}|X|}
            \hline
            \verb|In|
            & 
            \verb|C2C2 = cartesian\_product([C2, C2])|
            \\
            \hline
        \end{tabularx}
            
        \begin{tabularx}{11.5cm}{|p{0.60cm}|X|}
            \hline
            \verb|In|
            & 
            \verb|t_C2C2 = C2C2.cayley_table().table()|
            \\
            \hline
        \end{tabularx}

        {\setlength{\arraycolsep}{2ex}
        \[\begin{array}{r|*{4}{r}}
            \multicolumn{1}{c|}{\ast}&a&b&c&d\\\hline
            {}a&a&b&c&d\\
            {}b&b&a&d&c\\
            {}c&c&d&a&b\\
            {}d&d&c&b&a\\
        \end{array}\]}

        On obtient bien deux tableaux différents, ce qui laisse supposer que $(Z/4Z, +, 0)$ n'est pas isomorphe a $(Z/2Z \times Z/2Z, +, 0)$.
        \pagebreak
        \subsection{Elément neutre}
        \textbf{Question 2 :}
        
        Procédure SageMath :

        \lstinputlisting[language=Python, firstline=1, lastline=8]{programs.py}

        \begin{tabularx}{11.5cm}{|p{0.60cm}|X|}
            \hline
            \verb|In|
            & 
            \verb|test_neutral(t_C4)|
            \\
            \hline
        \end{tabularx}

        \begin{tabularx}{11.5cm}{|p{0.60cm}|X|}
            \hline
            \verb|Out|
            & 
            \verb|0|
            \\
            \hline
        \end{tabularx} \newline
        
        La procédure renvoit bien l'indice du premier élément qui correspond à $()$ dans SageMath.\newline

        \begin{tabularx}{11.5cm}{|p{0.60cm}|X|}
            \hline
            \verb|In|
            & 
            \verb|test_neutral(t_C2C2)|
            \\
            \hline
        \end{tabularx}

        \begin{tabularx}{11.5cm}{|p{0.60cm}|X|}
            \hline
            \verb|Out|
            & 
            \verb|0|
            \\
            \hline
        \end{tabularx} \newline

        La procédure renvoit bien l'indice du premier élément qui correspond à $((), ())$ dans SageMath. \newline

        \begin{tabularx}{11.5cm}{|p{0.60cm}|X|}
            \hline
            \verb|In|
            & 
            \verb|t_random = |
            \\
            \verb||
            & 
            \verb|[0, 2, 1, 3], [1, 3, 0, 1], [2, 1, 3, 0], [3, 1, 2, 0]]|
            \\
            \hline
        \end{tabularx}

        \begin{tabularx}{11.5cm}{|p{0.60cm}|X|}
            \hline
            \verb|In|
            & 
            \verb|test_neutral(t_random)|
            \\
            \hline
        \end{tabularx}

        \begin{tabularx}{11.5cm}{|p{0.60cm}|X|}
            \hline
            \verb|Out|
            & 
            \verb||
            \\
            \hline
        \end{tabularx}\newline

        La procédure ne renvoit rien.
        print C4.identity()
        \pagebreak
        \subsection{Elément symétrique}
        \textbf{Question 3 :}

        Procédure SageMath :

        \lstinputlisting[language=Python, firstline=10, lastline=18]{programs.py}

        \begin{tabularx}{11.5cm}{|p{0.60cm}|X|}
            \hline
            \verb|In|
            & 
            \verb|symetric(t_C4)|
            \\
            \hline
        \end{tabularx}

        \begin{tabularx}{11.5cm}{|p{0.60cm}|X|}
            \hline
            \verb|Out|
            & 
            \verb|(0, 0) (0, 1) (0, 2) (0, 3) (1, 0) (1, 1) (1, 2) (1, 3)|
            \\
            \verb||
            &
            \verb|(2, 0) (2, 1) (2, 2) (2, 3) (3, 0) (3, 1) (3, 2) (3, 3)|
            \\
            \hline
        \end{tabularx}\newline
        
        L'opération $+$ étant commutative, la procédure renvoit tous les couples possibles.\newline

        \begin{tabularx}{11.5cm}{|p{0.60cm}|X|}
            \hline
            \verb|In|
            & 
            \verb|symetric(t_C2C2)|
            \\
            \hline
        \end{tabularx}

        \begin{tabularx}{11.5cm}{|p{0.60cm}|X|}
            \hline
            \verb|Out|
            & 
            \verb|(0, 0) (0, 1) (0, 2) (0, 3) (1, 0) (1, 1) (1, 2) (1, 3)|
            \\
            \verb||
            &
            \verb|(2, 0) (2, 1) (2, 2) (2, 3) (3, 0) (3, 1) (3, 2) (3, 3)|
            \\
            \hline
        \end{tabularx}\newline
        
        De même, on obtient le même résultat avec $(Z/2Z \times Z/2Z, +, 0)$. \newline


        \subsection{Associativité}
        \textbf{Question 4 :}
    \section{Test d'un morphisme}
    \textbf{Question 5 :}

    1.

    2.

    3.

    4.
    \section{Où l’on identifie tous les groupes d’ordre 4}
    \textbf{Question 6 :}

    Supposons que un élément apparaisse deux fois dans la même ligne ou colonne, on a alors :
    \[(\exists a, b, c \in E) \quad a*b=a*c \Longrightarrow b = c\]
    Cela suppose donc deux éléments identiques.

    \textbf{Question 7 :}

    1. La condition sur l'élément neutre est vérifiée par construction, il reste à vérifier les conditions d'associativité et de symétrie.

    2.
    \section{Recherche et exploration avec SageMath}
    \textbf{Question 8 :}

    \textbf{Question 9 :}

    \textbf{Question 10 :}


\end{document}
